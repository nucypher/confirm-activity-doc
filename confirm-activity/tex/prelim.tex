\section{Preliminaries}
\label{prelim}
This section provides an overview of several concepts used in the proposed solutions. These include the framework of proof carrying data (PCD), and the collective signing (CoSi) protocol.


\subsection{Proof Carrying Data (PCD)}
The paradigm of PCD~\cite{chiesa2010proof} allows proving the correctness of a distributed computation performed by a set of untrusted parties. PCD produces a single proof that attests not only to the correctness of the final result, but also to the correctness of the entire history of intermediate computations that produced this result. Correctness here is defined as a polynomially computable predicate that abstracts the properties, or invariants, that must be satisfied.


\begin{figure}[h!]
\centerline{
\includegraphics[height= 1.3in, width = 1.0\columnwidth]{figures/pcd-diagram.pdf}}
\caption{An example of a PCD application. A distributed computation involving parties $P_i$, for 
$i \in \{1, \dots, 6\}$, each 
of which has a local input $x_i$ and produces an output $y_i$ with a proof $\pi_i$. }
\label{pcd-diagram}
\end{figure}


To clarify how PCD works, consider a distributed computation task that involves the set of parties shown in Figure~\ref{pcd-diagram}. Party $P_1$ starts the computation with its own input, produces an intermediate output with a proof saying that it performed the computation as defined by the protocol. It then sends both the output and the proof to the next party, which is $P_2$ in this case. Here, $P_2$ will have its own input $x_2$, as well as everything it received from $P_1$ (including the proof) as inputs to the next step of the computation. Similarly, $P_2$ will produce another intermediate output and a proof. This proof not only attests to the correctness of the computation done by $P_2$, but also to the history that led to the result $y_2$. In other 
words, it implicitly includes, or accumulates, the proof from $P_1$. The same process continues until the full computation is finished. Anyone can verify the correctness or compliance of the whole distributed protocol by only verifying the last proof $\pi_6$ produced by the exit party, which is $P_6$ in the figure.


As shown, PCDs enable untrusted parties to work with each other in a fully distributed fashion, without overwhelming the system with the storage and verification of individual proofs for each step of the distributed computation, and without re-executing any of the intermediate computations. 


\subsection{Collective Signing (CoSi)}
\label{cosi}
CoSi~\cite{syta2016keeping} is a protocol that enables a set of parties to cosign a statement together while producing a single signature. Thus, both the verification time and the space requirements are just like having a single signer. However, this signature holds all parties accountable for the statement they agreed to sign.


CoSi builds upon Schnorr multi-signatures~\cite{schnorr1991efficient, bellare2006multi, micali2001accountable}, but combines them with communication trees to speed up the signing process, especially when having a large number of cosigners. In what follows, we only present the protocol with a plain communication architecture as it suffices for the confirm service activity solution introduced in this document.


Schnorr signatures work in a group $\mathbb{G}$ of a prime order $q$ and generator $g$, such that the discrete log is believed to be hard in this group. Take $n$ parties that want to sign a statement together. Each of these parties has a secret $sk_i \in \mathbb{Z}_q$ and a public key $pk_i \in \mathbb{G}$ such that $pk_i = g^{sk_i}$. To sign a message $m$, one of these parties, say $P_1$, coordinates the signing process as follows: 
\begin{enumerate}
\setlength{\itemsep}{0pt}
\item $P_1$ prepares a message $m$ and sends it to the rest of the signers.

\item Each party $P_i$, possibly after verifying $m$, selects a secret $\tau_i \in \mathbb{Z}_q$ and computes $V_i = g^{\tau_i}$, then it sends $V_i$ back to $P_1$.

\item $P_1$ aggregates all random values received from the signers, and its own, by computing $V = \Pi_{i =1}^n V_i$.

\item $P_1$ then computes $c = H(V||m)$, where $H$ is an appropriate hash function and $||$ is a concatenation operator. $P_1$ then sends $c$ to the rest of the parties.

\item Each party computes a response $r_i = \tau_i - sk_i\cdot c$ and sends it to $P_1$.

\item Lastly, $P_1$ computes $r = \sum_{i=1}^n r_i$, and outputs the collective signature over $m$ as $(c, r)$.
\end{enumerate}


Verifying the signature proceeds as in classical Schnorr signatures~\cite{schnorr1991efficient} with one difference. An aggregated public key $pk$ is used in the verification process, which is computed as $pk = \Pi_{i=1}^{n} pk_i$.


The work in~\cite{syta2016keeping} tackles several issues related to the availability of the signing parties, and optimizing the communication between them. We believe that such techniques can be used in the NuCypher network if we adopt the CoSi-based solution for the confirm service activity issue.