\section{Threat and Network Models}
\label{threat-network-model}
This section describes a modification for the network model of NuCypher required by the proposed solutions, in 
addition to the threat model adopted in this work.


\subsection{Network Model}
In order for the proposed solutions to work, we need to ensure freshness, integrity, and authenticity of the requests (as well as service complaints as we will see later) issued by Bob. This can be achieved by requiring Bob to sign each re-encryption request it issues and to include a timestamp, or sequence number, in each request to ensure freshness. The keypair used for the signature should be separate from the keypair Bob uses for proxy encryption in the system. 


Until now we assume that Alice pays for the service by using the Ether she locks in the policy contract. Other arrangement may emerge in the system like having Bob pay for each requests he issues. Such an arrangement and its implication on the confirm activity issue will be studied once it becomes part of the NuCypher network protocol.


\subsection{Threat Model}
In our threat model, we make the following set of assumptions:
\begin{itemize}
\item {\bf Self-interested parties:} We do not place trust in any party and we assume that all participants are 
self-interested. This means that a party may decide to follow the protocol or deviate from it, either on its own or by colluding with other attackers, and such decision is solely based on what maximizes the financial profits
of this party.

\item {\bf Attackers' collusion:} If it is profitable, an Ursula may, for example, spin out  its own Alice and/or Bob or collude with Alice. We do not consider collusion between Bob and Ursula as a practical threat. This means that cases of Bob pretending  that he got service from Ursula (while no service is delivered) is not an issue. We do not 
suspect this will be of any importance and there is no motivation to do it given that the work Ursula does for re-encryption is minimal.

The above non-collusion assumption does not affect the solutions proposed to handle the confirm service activity discussed in Section~\ref{confirm-service-activity}. This means that these solutions provide higher security guarantees and address the issue of potential collusion between Bob and Ursula. It could be the case that relaxing the requirement of addressing this collusion case allows deploying more efficient solutions. In order to make an educated decision of the plausibility of this threat, we need more data about the system operation and the behavior of the participants once the inflation rewards disappears in the system. Therefore, we leave this 
issue until later when fees become the only source of rewards for Ursulas.

\item {\bf Honest Ursulas:} We also assume that when sampling a subset of Ursulas in the network, at least one of them is honest. This assumption can be achieved by having a global assumption regarding the lower bound of the number of honest Ursulas with respect to the total number of Ursulas in the system. (Or it can be achieved by deploying a special entity like an external verifier, that changes on a periodic basis, or one by the NuCypher company, that is trusted to faithfully participate as a member of each samples set. This verifier does not provide any other services in the system.)
\end{itemize}


Other than the above, we have usual assumptions like dealing with computationally bounded adversaries 
that cannot break secure cryptographic primitives with non-negligible probability. (We may need to work in the random oracle model, but this depends on the security requirements of the proposed solutions.)



