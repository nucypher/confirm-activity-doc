\section{Confirm Availability Solution}
\label{confirm-availability}
In this section, we present a solution for confirming availability of Ursulas, which will be used during early operation stage of the network. The proposed solution is called \emph{Optimistic Challenge-based Approach}. It is centered around two important concepts: requiring minimal changes to the current network implementation, and minimizing the additional overhead. We introduce the design details of our solution, after which we discuss some potential security attacks and how to address them.


\paragraph{\bf Overview.} The optimistic challenge-based approach is based on a detect-and-punish technique. As long as everything work as expected, no extra measures will be invoked. In other words, investigating the status of a specific Ursula is activated only when Bob does not hear back regarding a re-encryption request he has issued. Thus, Bob proceeds as usual in requesting the service. If any of the contacted Ursulas does not reply, Bob submits a complaint claiming that this Ursula is unavailable. This complaint triggers the challenge phase in the network to verify Bob's claim, and hence, punish Ursula financially (by slashing part of her stake) in case the claim found to be valid.


The proposed scheme labels Ursulas in the network under two classes: Working Ursulas, which is the conventional role of maintaining access policies and providing the re-encryption service. And gateway Ursulas, which is an additional role of handling Bob's complaints. Gateway Ursulas are a subset of the working Ursulas that is selected at random for each round.\footnote{A round is the time needed to mine a block or several blocks on the blockchain.} Hence, each working Ursula will get to serve as a gateway from time to time. 


Accordingly, Bob contacts working Ursulas listed in Alice's policy. If any of them does not reply, Bob reaches out to a gateway Ursula and forwards her the request. The gateway in turn mediates the service session by forwarding the request to the working Ursula again and waiting for a reply. If a reply is received, the gateway sends it to Bob and closes the case. 


On the other hand, if no reply is received, the gateway notifies other gateways selected for the current round to repeat the mediation process. If none of them receive a reply, they collectively sign (using the CoSi protocol described in Section~\ref{cosi}) a statement against the working Ursula, which we call a proof-of-unavailability. This proof is submitted to the \stakeescrow contract, and after being verified, part of the working Ursula's stake will be slashed as a punishment. 


\paragraph{\bf Working Ursulas discovery.} 
In the NuCypher network there is a separation between the role of a staker and a working Ursula. All stakers' Ethereum addresses, along with their stake values, are recorded in one of the NuCypher network smart contracts, namely, the \stakeescrow contract. Each of these stakers will be tied, or bonded, to a working Ursula to provide the re-encryption service by having its address listed next to the staker's address. 


For each access policy, Alice delegates the re-encryption service to a designated set of working Ursulas. Each of these Ursulas will be supplied with a re-encryption key fragment that converts a ciphertext encrypted under Alice's public key into a ciphertext fragment encrypted under Bob's public key. Once Bob gets a sufficient number of ciphretext fragment, based on some threshold, he can combine these fragments together to get the complete ciphertext, which he can decrypt. 


The \stakeescrow contract does not contain any information about how to reach a specific working Ursula. Thus, and as part of the access policy, Bob receives a map from Alice containing the set of working Ursulas that has the key fragments, along with their Ethereum addresses. To allow the parties to communicate with each other, each participant, i.e., Alice, Bob, Ursula, employs a discovery protocol (more like a gossiping protocol) to discover the IP addresses and ports of the working Ursulas in the network. Hence, Bob uses the map (in association with the network discovery protocol) to reach each of these Ursulas and obtain the service. 


The discovery process also includes retrieving the keypair a working Ursula uses for the re-encryption service purposes. We call this keypair a stamp, and we require an Ursula to use her stamp when signing all messages related to the proposed confirm availability protocol. 


\paragraph{\bf Gateway Ursulas selection.} 
For each round a set of $n$ gateway Ursulas will be selected. A proof-of-unavailability will be accepted by the network if it is signed by at least $t$ gateways among this set (this threshold is a relaxation of requiring all gateways to sign since it could be the case that not all of them are active). 


A simple idea to select the gateways, similar to the deterministic lottery protocol found in~\cite{almashaqbeh2020microcash}, is to use a high entropy source of randomness to obtain a random beacon for each round. Then feed this beacon into an iterative hashing process. That is, let the random beacon of the $i^{th}$ round be $r_i$, and let a hash function be denoted as $H(\cdot)$. To select the first gateway, the hash $h_1 = H(r_i)$ is computed, and then it is mapped to a working Ursula index as listed in the \stakeescrow contract. That is, the working Ursula bonded with the first staker listed in the contract has index 0, the next has index 1, etc. Then, to select the second gateway, hash the previous hash to obtain $h_2 = H(h_1)$ and map the output to a working Ursula index, and so no. In case of a collision, i.e., an already selected index is selected again, discard it and proceed to the next hash iteration. This process continues until a set of distinct $n$ indices is computed. 


It should be noted that since the random beacon is public, although it will be known only when round $i$ starts, any party can compute the set of selected indices. Thus, Bob performs the hash iterative process when needed to obtain the gateways. The selection can be also verified by repeating the hash iterative process by any other party.


Furthermore, note that the selection process may produce Ursulas for which Bob does not have a contact information. Hence, Bob has to discover these Ursulas before being able to submit a complaint. To reduce the delays, and as outlined earlier, Bob starts with complaining to one gateway and involves the rest if the working Ursula does not respond to the challenge from this gateway. Hence, Bob can start with a gateway that he already knows how to reach (if any) and in the meantime works on discovering the rest of the selected gateways (if he does not already have their contact information).


The above protocol assumes working in the random oracle model, meaning that hash functions are modeled as random oracles. Thus, the iterative hashing process is conjectured as producing a random, unbiased output. Although, we believe this is sufficient for our purposes, this can be replaced with more sophisticated approaches to produce random strings (that are then mapped to indices), e.g., a verifiable random function~\cite{micali1999verifiable}. 


As for the high entropy source of randomness, there are several potential approaches to fulfill it. We can rely on an external source, e.g., the NIST random beacon~\cite{nist-beacon}, or even a combination of multiple sources. Or we can also resort to the random oracle model and assume that block hashes are drawn from a uniform distribution, and thus, use the hash of the block mined in the current round as a random beacon.\footnote{To handle inconsistencies of the blockchain view, the hash of the block that is confirmed at the current round can ebe used. This is the block that was mined $y$ rounds ago where $y$ is the confirmation interval in the underlying cryptocurrency system.} Alternatively, and to address concerns regarding the low entropy of block hashes, we can use randomness extractors to produce a high entropy beacon from the block hashes~\cite{bonneau2015bitcoin}.


\paragraph{\bf Proof-of-unavailability processing.} 
As mentioned previously, a proof-of-unavailability against a working Ursula is a statement stating that this Ursula was unavailable during a specific round signed by at least $t$ gateways out of the $n$ gateways selected for that round. Such a proof is submitted to the \stakeescrow contract and is verified as follows:
\begin{itemize}
\item Reproduce the list of gateways selected for the current round, by using the same process described previously, to verify the signing gateways are legitimate.

\item Compute the collective verification key of the cosigners and verify the collective signature over the signed statement (see Section~\ref{cosi}).

\item If all checks pass, meaning that the proof-of-unavailability is valid, the \stakeescrow contract revokes part of the stake bonded to the working Ursula as a punishment.
\end{itemize}


Note that the above assumes that the \stakeescrow contract is aware of the gateways' stamps (or keys) that are used in signing the proof-of-unavailability. This can be done by either expanding the staker list in the contract to contain not only the Ethereum addresses of working Ursulas, but also their stamps. Another option is to piggyback the stamps (signed by using the key corresponding to the working Ursula's Ethereum address) with the proof-of-unavailability. A more simpler, and efficient, option is to sign the proof-of-unavailability by using the key corresponding to the working Ursula's Ethereum address in the first place, and thus, the \stakeescrow contract will not need an additional information. Which option to use depends merely on efficiency considerations.


\paragraph{\bf Working only during the challenge phase.}
A potential attack on our scheme is that a working Ursula may operate only during the challenge phase. That is, she ignores any fresh request coming from Bob, and just replies afterwards when receiving the same request again (i.e., when it is being forwarded by the gateways during the challenge phase). 


We argue that such an attack is not a practical issue since a rational Ursula will be online eitherway to monitor whether this is a challenge phase or not. Also, the amount of work needed to respond to a re-encryption request is minimal. Hence, there is no incentive at all to follow this strategy, meaning that a rational Ursula will act honestly.


\paragraph{\bf DoS attacks.} 
Another potential issue is exploiting the unavailability complaints to perform a DoS attack against the gateways and working Ursulas. In detail, Bob may issue fake complaints against working Ursulas, even though they are responsive. Alternatively, an attacker (insider or outsider) may replay these complaints, or issue them on behalf of Bob. The goal could be to slash Ursula's stake or to overwhelm the gateways with the challenge process and make them unavailable to perform the re-encryption service (recall that the gateways are originally working Ursulas in the NuCypher network).


This issue can be handled using a collection of techniques as follows:
\begin{itemize}
\item Require Bob to sign all the complaints (so no one can impersonate Bob unless they know his signing key), and include a timestamp or a sequence number in each of them (to ensure freshness and prevent replay attacks).

\item Increase the cost of issuing complaints. That is, require Bob to do some work in order to produce a valid complaint (in a similar way to proof-of-work, i.e., produce a hash over the complaint with specific number of leading zeros).
\end{itemize}


Furthermore, since gateways Ursulas are selected fresh at random for each round, an Ursula will get the chance to be only a working Ursula during some rounds. This reduces the impact of gateway DoS attacks for any Ursula as the gateway role is rotating and not fixed.


\paragraph{\bf Unresponsive gateways.}
There is a chance that the gateways themselves could be unresponsive. This will stall the challenge phase and make it infeasible to prove that some working Ursula is unavailable. We believe that such an issue will be organically handled by the network. That is, the larger the number of Ursulas in the network, the higher the chances of having more honest Ursulas. By increasing the number of selected gateways $n$, and given that the selection process is random, the chances of having at least $t$ honest gateways in a round will be high.


Moreover, to encourage gateway participation, a potential approach is to incentivize Ursulas to do the work by paying them a small fee. Such a fee could be out of the revoked stake (if any) or paid by Alice from her access policy funds.


It could be the case that all gateways in a round are unresponsive (although we believe that the probability this happens is very low). In this case, it could be viable to perform the challenge phase in a recursive way. In other words, Bob can send an unavailability complaint against the gateways of the current round to the gateways of the previous round. The previous round gateways will in turn challenge the current round gateways by forwarding Bob's complaint to them and sign a proof-of-unavailability if no response is received. 


\paragraph{\bf Quantifying the financial punishment.}
A question that comes to mind is how much should be revoked from Ursula's stake once a proof-of-unavailability is approved? Quantifying this amount should be driven by two factors; the unavailability duration and the utility gain (or profits) of this cheating behavior in terms of the collected fees and subsides during that period.


We can compute this value to be equal to the fees and subsidies that an Ursula is supposed to receive in a round. Hence, the working Ursula will, first, lose her round payments from the network (she will not get paid by Alice or by the network), and second, lose the same amount from her stake. Such a loss will take place even if a single proof-of-unavailability is received against a working Ursula within a round. In fact, the \stakeescrow contract will process only one proof against a working Ursula per round to reduce computational costs in the system. 

